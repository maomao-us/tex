\documentclass[technicalreport]{ieicej}
%\documentclass[technicalreport,usejistfm]{ieicej}
\usepackage[dvips]{graphicx}
\usepackage{latexsym}
\usepackage[fleqn]{amsmath}
\usepackage[psamsfonts]{amssymb}

\def\IEICEJcls{\texttt{ieicej.cls}}
\def\IEICEver{1.3}
\newcommand{\AmSLaTeX}{%
 $\mathcal A$\lower.4ex\hbox{$\!\mathcal M\!$}$\mathcal S$-\LaTeX}
\newcommand{\PS}{{\scshape Post\-Script}}
\def\BibTeX{{\rmfamily B\kern-.05em{\scshape i\kern-.025em b}\kern-.08em
 T\kern-.1667em\lower.7ex\hbox{E}\kern-.125em X}}

%%%%%%%%%%%%%%
\small
\jtitle{受動歩行における歩行パターンの動力学}
%%\jsubtitle{技術研究報告原稿のための解説とテンプレート}
%%%%%%%%%%%%%%%%%%%%%%%%%%%%%%%%%%%%%%%%%%%%%%%%%%%%%%%%%%%%
%%
%%  これは国際会議ね
%%\etitle{Passive Dynamical Walking and Coexistance of
%%        Different Walking Pattern Attractors}
%%%%%%%%%%%%%%%%%%%%%%%%%%%%%%%%%%%%%%%%%%%%%%%%%%%%%%%%%%%%
%%
%% これは英語の本当の論文用かな
%%\etitle{Different Attractors Coexistance of
%%        Walking Pattern in Passive Dynamical Walking}
%%%%%%%%%%%%%%%%%%%%%%%%%%%%%%%%%%%%%%%%%%%%%%%%%%%%%%%%%%%%
\etitle{Dynamical analysis of Passive Dynamic Walking}
%%\esubtitle{Guide to the Technical Report and Template}
\authorlist{%
  \authorentry[toyoda@i.his.fukui-u.ac.jp]{豊田 範哲}{Noriaki TOYODA}{Fukui1}
  \authorentry[jou@i.his.fukui-u.ac.jp]{黒岩 丈介}{Jousuke KUROIWA}{Fukui1}
  \authorentry{小倉 久和}{Hisakazu Ogura}{Fukui1}
  \authorentry{小高 知宏}{Tomohiro Odaka}{Fukui2}
  \authorentry{高橋 勇}{Isamu Takahashi}{Fukui1}
  \authorentry{白井 治彦}{Haruhiko Shirai}{Fukui3}
}
\affiliate[Fukui1]{福井大学大学院工学研究科知能システム工学専攻\hskip1zw
  〒910--0017 福井県福井市文京3--9--1}
{Department of Human \& Artificial Intelligent Systems, Graduate
  Course of Engineering, Graduate School, University of Fukui,
  Bunkyou 3--9--1, Fukui-city, Fukui,
  910--0017 Japan}
\affiliate[Fukui2]{福井大学大学院原子力・エネルギー安全工学専攻\hskip1zw
  〒910--0017 福井県福井市文京3--9--1}
  {Graduate Program 0f Nuclear Power and Energy Safety
  Engineering,University of Fukui}
\affiliate[Fukui3]{福井大学工学部技術部\hskip1zw
  〒910--0017 福井県福井市文京3--9--1}
  {Department of Engineering,Department of Technology,University of Fukui}

\MailAddress{$\dagger$\{taro,jiro\}@i.his.fukui-u.ac.jp}

%%%%%%%%%%%%%%%%%%%%%%%%%%%%%%%%%%%%%%%%%%%%%%%%%%%%%%%%%%
%
%  概要(日本語)
%
%%%%%%%%%%%%%%%%%%%%%%%%%%%%%%%%%%%%%%%%%%%%%%%%%%%%%%%%%%
\begin{document}
\begin{jabstract}
本研究では,2リンク受動歩行にあらわれる歩行パターンを,数値的に明らかにすることを目的としてい
る.このシステムでは,床の傾斜角度及び腰と足の質量比が重要なシステムパラメータとなる.従来の研究では,足
の質量が腰の質量に対して十分に小さいと仮定していた.この仮定によって,初期値のパラメータ空間を4次元から
2次元に縮約していた.この仮定のもと,特定の初期値から歩行を開始すると,歩行パターンは床の傾斜角度と脚の
長さをシステムパラメータとして,唯一の歩行パターンに収束していた.本研究では,仮定を用いずに全ての歩行パ
ターンを数値的に解析した.その結果,同時に2つの異なる歩行パターンのアトラクタが共存することが分かった.
\end{jabstract}
\begin{jkeyword}
受動歩行,歩行パターン,アトラクタ共存
\end{jkeyword}
%%%%%%%%%%%%%%%%%%%%%%%%%%%%%%%%%%%%%%%%%%%%%%%%%%%%%%%%%%
%
%  概要(英語)
%
%%%%%%%%%%%%%%%%%%%%%%%%%%%%%%%%%%%%%%%%%%%%%%%%%%%%%%%%%%
\begin{eabstract}
The purpose of the paper is to investigate all possible walking patterns in 2-link passive dynamical
walking. In this system, the slope angle and the ratio of the hip's mass and the leg's mass are important system
parameters. In the conventional study, it is assumed that the mass of the hip is much larger than the center of
mass of the leg. This assumption gives a big merit that the initial conditionspace shrinks to two dimension from
four. Based on the assumption, starting from a certain initial condition, the walking pattern certainly converges to
unique attractor depending on the system parameters, the slope angle and the leg length. In the paper, we analyze
all the possible walking patterns numerically without the assumption. The results show that there coexist different
two attractors of walking pattern simultaneously.
\end{eabstract}
\begin{ekeyword}
passive dynamic walking,
walking pattern, attractor coexistance
\end{ekeyword}
\maketitle
\newblock
\newpage
\newblock
\newpage

%%%%%%%%%%%%%%%%%%%%%%%%%%%%%%%%%%%%%%%%%%%%%%%%%%%%%%%%%%
%
%  まえがき
%
%%%%%%%%%%%%%%%%%%%%%%%%%%%%%%%%%%%%%%%%%%%%%%%%%%%%%%%%%%
\section{まえがき}


%%%%%%%%%%%%%%%%%%%%%%%%%%%%%%%%%%%%%%%%%%%%%%%%%%%%%%%%%%
%
%  第1章
%
%%%%%%%%%%%%%%%%%%%%%%%%%%%%%%%%%%%%%%%%%%%%%%%%%%%%%%%%%%
%\section{2 リンク受動歩行の定式化}
本研究で用いる2リンク受動歩行ロボットのモデルは,
図 \ref{fig:model} に示すような2次元のモデルを考える\cite{4}.
これは人の下半身をモデル化したものである.
脚は質量のない剛体と脚先からaの距離にある質点で表わす.
ここでは,接地している方の脚を支持脚,もう一方の脚を遊脚と呼ぶことにする.
腰は2つの脚を連結する大きさの無視できるリンクとし,
その質量は上半身を代表するものと考える.
床面は脚と非弾性衝突を起こし,摩擦は無限大であるとする.
そのため,支持脚と床の接点はリンクと考えることができる.
また,各剛体の回転による慣性モーメントと関節の粘性摩擦は無視する.
%
% 図 %%%%%%%%%%%%%%%%
\begin{figure}[tb]
  \begin{center}
    \includegraphics[width=5cm]{./images/model.eps}
%    \vspace{45mm}
    \caption{2リンク受動歩行ロボットのモデル}
    \label{fig:model}
    \ecaption{A model of two link Passive Dynamic Walking robot}
  \end{center}
\end{figure}
%%%%%%%%%%%%%%%%%%%%%
%
図\ref{fig:model}中,また以降の方程式で扱う記号は表\ref{table:kigou}に示す通りである.

% 表 %%%%%%%%%%%%%%%%%
\begin{table}[tb]
\caption{記号の説明}
\label{table:kigou}
\ecaption{Explanation of a sign}
\begin{center}
  \begin{tabular}{c|c}
    \Hline
    記号 & 記号説明 \\ \hline
    $\theta_1$ & 床面と垂直な方向を基準とした支持脚の角度\\
    $\theta_2$ & 床面と垂直な方向を基準とした遊脚の角度\\
    $\phi$ & 水平方向からの床の傾斜角度\\
    $m_{\mathrm{H}}$ & 腰の質量\\
    $m$ & 足の質量\\
    $\beta$ & \FRAC{$m$}{$m_h$}\\
    $a$ & 腰から足の質点までの距離\\
    $b$ & 足先から足の質点までの距離\\
    $l$ & 足の長さ\\
    $\mu$ & \FRAC{$a$}{$l$}\\
    $\nu$ & \FRAC{$b$}{$l$}\\
    $\mathrm{A},\mathrm{B}$ & 支持脚,遊脚の質点の位置\\
    $\mathrm{C},\mathrm{D}$ & 支持脚,遊脚の足先の位置\\
    $\mathrm{H}$ & 腰の位置\\
    $S$ & $\sin(\theta_1 - \theta_2)$\\
    $C$ & $\cos(\theta_1 - \theta_2)$\\
    \Hline
  \end{tabular}
\end{center}
\end{table}
%%%%%%%%%%%%%%%%%%

%%%%%%%%%%%%%%%%%%%%%%%%%%%%%%%%%%%%%%%%%%%%%%%%%%%%%%%%%%
\begin{thebibliography}{99}
\bibitem{3}
  李清華,高西淳夫,加藤一郎.
  ZMPを安定規範とした2足歩行ロボットの上体補償運動の学習制御.
  日本ロボット学会誌,Vol.11,No.4,pp.557-563,1993.

\end{thebibliography}

%%%%%%%%%%%%%%%%%%%%%%%%%%%%%%%%%%%%%%%%%%%%%%%%%%%%%%%%%%
%
%  付録
%
%%%%%%%%%%%%%%%%%%%%%%%%%%%%%%%%%%%%%%%%%%%%%%%%%%%%%%%%%%
\appendix
%%%%%%%%%%%%%%%%%%%%%%%%%%%%%%%%%%%%%%%%%%%%%%%%%%%%%%%%%%
%  運動方程式
%%%%%%%%%%%%%%%%%%%%%%%%%%%%%%%%%%%%%%%%%%%%%%%%%%%%%%%%%%
\section{運動方程式}

\subsection{振子運動時の方程式}
\label{sec:undou}

全体の運動エネルギーを$T$とすると
\begin{eqnarray}
    T &=& \frac{1}{2}  m(\dot{x}_A^2 + \dot{y}_A^2)
       + \frac{1}{2}m_H(\dot{x}_H^2 + \dot{y}_H^2)
       + \frac{1}{2}  m(\dot{x}_B^2 + \dot{y}_B^2)\nonumber\\
      &=& \frac{1}{2}(ma^2 + m_Hl^2 + ml^2)\dot{\theta}_1^2\nonumber\\
      & & \mbox{} + \frac{1}{2}mb^2\dot{\theta}_2^2
       - \frac{1}{2}mlb\dot{\theta}_1\dot{\theta}_2C
\end{eqnarray}
である。

\end{document}
