\input{texheder.tex}
\usepackage{setspace} % setspaceパッケージのインクルード
\usepackage{enumitem}
\usepackage{grffile}
\usepackage{url}
\usepackage{diagbox}

%「Weekly Report」
\newcommand{\Weekly}[5]{
\twocolumn[
    \begin{center}
    \bf
    第 #1 回 Weekly Report\\
    \huge
    スマートフォンのスイング動作に潜在する個人特性\\
    \LARGE
    --- フルフルによる個人認証 ---\\

    \end{center}
    \begin{flushright}
        #2 月\ \ \  #3 日 \ \ \ #4 \
        #5
    \end{flushright}
    ]
}
%\setstretch{0.5} % ページ全体の行間を設定

\begin{document}

    \Weekly{5}{5}{11}{(火)}{\ 中島 基晴}

    % 1章
    \section{今週までの作業内容}

        \begin{itemize}
            \item 計測データの収集
            \item 極値の抽出
            \item 検定
        \end{itemize}

    % 2章
    \section{極値を抽出する際のルール}
        極値を抽出する時,グラフの極大値か極小値どちらを
        抽出するのかを決定しなければならない.
        ここでスマートフォンの加速度センサーはすべて
        次の図\ref{fig:direction}のように設定されている.
        %%%%%%%%%%%%%%%%%%%%%%%%%%%%%%%%%%%%%%%%%%%%%%%%%%%%%%%%%%%%%%%
        \begin{figure}[h]
            % \centering
            \vspace{50mm}
            \hspace{1.5cm}
            \includegraphics[scale=0.6]{images_S5/sumaho.eps}
            % \hspace{-10.0cm}
            \caption{スマートフォンの加速度の方向}
            \label{fig:direction}
        \end{figure}
        %%%%%%%%%%%%%%%%%%%%%%%%%%%%%%%%%%%%%%%%%%%%%%%%%%%%%%%%%%%%%%%
        仮に右手にスマートフォンを持ってスイング動作を行い場合,
        スマートフォンの画面は顔側にあるので振り切った時の極値は
        x軸方向,y軸方向ともにマイナス方向に向かって大きくなるので
        極小値になるはずである.
        そして次の図\ref{fig:graph1}と図\ref{fig:graph2}が
        スイング動作の結果である.
        図\ref{fig:graph1}と図\ref{fig:graph2}のグラフは仮称Aさんと仮称Bさんの
        グラフである.
        %%%%%%%%%%%%%%%%%%%%%%%%%%%%%%%%%%%%%%%%%%%%%%%%%%%%%%%%%%%%%%%
        \begin{figure}[h]
            % \centering
            \vspace{45mm}
            \hspace{-1cm}
            \includegraphics[scale=0.2]{images_S5/sintyoku_51.eps}
            % \hspace{8.0cm}
            \vspace{-0.5cm}
            \caption{Aさんのグラフ}
            \label{fig:graph1}
        \end{figure}
        %%%%%%%%%%%%%%%%%%%%%%%%%%%%%%%%%%%%%%%%%%%%%%%%%%%%%%%%%%%%%%%
        %%%%%%%%%%%%%%%%%%%%%%%%%%%%%%%%%%%%%%%%%%%%%%%%%%%%%%%%%%%%%%%
        \begin{figure}[h]
            % \centering
            \vspace{45mm}
            \hspace{-1cm}
            \includegraphics[scale=0.2]{images_S5/sintyoku_52.eps}
            % \hspace{3.0cm}
            \vspace{-0.5cm}
            \caption{Bさんのグラフ}
            \label{fig:graph2}
        \end{figure}
        %%%%%%%%%%%%%%%%%%%%%%%%%%%%%%%%%%%%%%%%%%%%%%%%%%%%%%%%%%%%%%%
        予想通り開始からx軸方向,y軸方向ともにマイナス方向に加速度が
        大きくなっている.
        Bさんのグラフは開始にプラス方向に向かっているが,これは静止状態から
        動き出すために一度腕を後ろに動かしたからと思われる.そのため
        その後のスマホを手元に戻す動作によって生じた極大値の値よりも小さい.
        ここで問題がz軸方向である.
        Aさんのグラフはx軸方向,y軸方向,z軸方向のすべての値が同じ方向に動いているが,
        Bさんのグラフはz軸方向だけ他の2つと逆である.
        この問題に対して,スイング動作で振り切った時に大半の人は手のひらが上に
        なるように動かすと思われる.
        つまりz軸のマイナス方向に動く.
        実際に他の人のデータではAさんのグラフのようになっているものが多かった.
        よってz軸方向も極小値を抽出することにする.

        極小値を抽出する時,まず単純移動平均法のようにあるステップのデータを中心に
        そのデータと前後n個のデータを取り出す.
        その後昇順にして一番値が小さいデータとと取り出す時に中心としたデータが
        一致していればそれが極小値となる.
        現在は前後30個のデータを取り出すように設定しているが,スイング動作が
        速い人にこの範囲で抽出を行うとほしいデータを取りこぼす可能性がある.
        よってこの範囲をどれほどにするのかを今後検討する.
        また,スイング動作を行う際に数秒待ってもらうがこのときにも僅かであるが手は動いている.
        よって図\ref{fig:graph1}と図\ref{fig:graph2}のグラフの最後部分をもらえばわかるが
        小さい極小値になっているところがある.
        もしもこのまま抽出を行うと,この小さな極小値も抽出してしまう.
        よって抽出の際は閾値を設けてその値を $-2$ とし,この値よりも小さい極小値を抽出する.
        絶対加速度の場合はx,y,zの絶対値が合わさったグラフになるので
        閾値を3倍の $6$ とする.

    % 3章
    \section{同じ人の計測データ同士の検定}

    本実験では違う人同士の計測データから有意差があるかどうかを調べるが,
    同じ人の計測データ同士にも有意差があるのかどうかも調べた.
    以下の表\ref{table:yuuisa}は現在計測を終えているデータを検定して
    有意差がなかったデータ,つまり違う人同士の計測データの検定で用いるデータ数を示す.
    時間の都合により計測したデータをすべて検定しておらず,またx軸方向だけ示す.
    %%%%%%%%%%%%%%%%%%%%%%%%%%%%%%%%%%%%%%%%%%%%%
    \begin{table}[h]
        \caption{同じ人の計測データ同士を検定した結果}
        \label{table:yuuisa}
        \vspace{0.5cm}
        \centering
        \begin{tabular}{|c|c|}
            \hline
            \diagbox{スイング数}{軸方向} & x \\\hline
            2回 & $15 〜 20$ \\
            3回 & $12 〜 20$ \\
            4回 & $15 〜 18$ \\
            5回 & $9 〜 19$ \\
            \hline
        \end{tabular}
    \end{table}

    結果は個人差によるである.
    1度に20回振ってもらったがほとんどのデータが研究に使える人もいれば,
    半分と少しのデータだけ使えるという人がいた.
    また,前半と後半でデータが変わったということもなく,最初から有意差が出る
    振り方をしていたりと様々である.

    5回のスイング動作だけ10回ずつの計測で行ったが,有意差があるデータが増え,
    余計に使えるデータ数が減った印象があった.
    10回の計測と20ぼ回計測どちらで行っても有意差があるデータ数は変わらない人もいたが,
    一人だけ1回目の10回計測データと2回目の10回計測データで違いがありすぎて
    どちらか一方のデータ群しか使えないということが起きた.
    実用化に向けてという意味ならば数回に分けてデータ収集することが現実に向き合っているが,
    データ収集をどのように行うかはこれからの課題である

    また,10回の計測に変更する場合,これまでに20回の計測で得たデータは使用できなくなるのかと
    疑問に思う.

    % 4章
    \section{今後の予定}
        他の座標軸でも有意差があると判断できるデータ数がどれほどあるのかを調べ,
        本研究の目的である個人間で有意差があるのかを調べる.
\end{document}