\input{texheder.tex}
\usepackage{setspace} % setspaceパッケージのインクルード
\usepackage{enumitem}
% \usepackage{graphicx}
\usepackage[dvipdfmx]{graphicx}
\usepackage{grffile}
\usepackage{url}
\usepackage{color}

%「Weekly Report」
\newcommand{\Weekly}[5]{
\twocolumn[
    \begin{center}
    \bf
    第 #1 回 Weekly Report\\
    \huge
    スマートフォンのスイング動作に潜在する個人特性\\
    \LARGE
    --- フルフルによる個人認証 ---\\

    \end{center}
    \begin{flushright}
        #2 月\ \ \  #3 日 \ \ \ #4 \
        #5
    \end{flushright}
    ]
}
%\setstretch{0.5} % ページ全体の行間を設定

\begin{document}

\Weekly{8}{6}{8}{(火)}{\ 中島 基晴}


%%%%%%%%%%%%%%%%%%%%%%%%%%%%%%%%%%%%%%%%%%%%%%%%%%%%%%%%%%%%%%%%%%%%%%%%%%%%%%%%%%%%%%%%%%%%%%%%%%%%%%%%%%%%%%%%%%
%%%%%%%%%%%%%%%%%%%%%%%%%%%%%%%%%%%%%%%%%%%%%%%%%%%%%%%%%%%%%%%%%%%%%%%%%%%%%%%%%%%%%%%%%%%%%%%%%%%%%%%%%%%%%%%%%%
\section{今週までの作業内容}
    \begin{itemize}
        \item 計測データの収集
        \item 検定
    \end{itemize}

%%%%%%%%%%%%%%%%%%%%%%%%%%%%%%%%%%%%%%%%%%%%%%%%%%%%%%%%%%%%%%%%%%%%%%%%%%%%%%%%%%%%%%%%%%%%%%%%%%%%%%%%%%%%%%%%%%
%%%%%%%%%%%%%%%%%%%%%%%%%%%%%%%%%%%%%%%%%%%%%%%%%%%%%%%%%%%%%%%%%%%%%%%%%%%%%%%%%%%%%%%%%%%%%%%%%%%%%%%%%%%%%%%%%%
\section{KP の質問の回答}

KP の時は質問に対して濁した回答をしてしまったのでここで改めて回答する.

\subsection{小松さんの質問に対する回答}

「ノイズによってできた極値は特徴量として取らないのか?」という質問の回答は取らないである.
理由は以下の図\ref{fig:noiz}をご覧頂きたい.
%%%%%%%%%%%%%%%%%%%%%%%%%%%%%%%%%%%%%%%%%%%%%
\begin{figure}[h]
    % \centering
    \vspace{35mm}
    \hspace{1.6cm}
    \includegraphics[scale=0.2]{images_S8/nakajima_takenaka.eps}
    \vspace{-5mm}
    % \hspace{-3.5cm}
    \caption{ある計測グラフと真似した時のグラフ}
    \label{fig:noiz}
\end{figure}
%%%%%%%%%%%%%%%%%%%%%%%%%%%%%%%%%%%%%%%%%%%%%
これは同じ人から撮ったデータである.
青いグラフはノイズがあるが黄色のグラフにはない.
つまりノイズが出るときと出ないときが存在する.
よって特徴量とするには不安定なのでノイズは特徴量とはしない.

\subsection{山名田さんの質問に対する回答}

「もしも振り方を真似しても有意差は出るの?」という質問に対する回答は
出るである.
ここでも小松さんの質問に対する回答のときのように以下の図\ref{fig:nakajima_takenaka}をご覧頂きたい.
%%%%%%%%%%%%%%%%%%%%%%%%%%%%%%%%%%%%%%%%%%%%%
\begin{figure}[h]
    % \centering
    \vspace{35mm}
    \hspace{1.6cm}
    \includegraphics[scale=0.2]{images_S8/noiz.eps}
    \vspace{-5mm}
    % \hspace{-3.5cm}
    \caption{ノイズのあるデータとないデータ(同人物)}
    \label{fig:nakajima_takenaka}
\end{figure}
%%%%%%%%%%%%%%%%%%%%%%%%%%%%%%%%%%%%%%%%%%%%%
これはある人に5回スイングを5回計測していただいた時のデータと,
その振り方を私が真似して計測したデータのグラフである.
自分の都合の良いように振り方を変えただろうという避難を避けるために全力で真似をした.
それは図\ref{fig:nakajima_takenaka}を見れば理解していただけると思う.
よく似ていると思わないだろうか?一見これでは有意差は出ないのではと思うかもしれない.
しかし結果は異なった.
%%%%%%%%%%%%%%%%%%%%%%%%%%%%%%%%%%%%%%%%%%%%%
\begin{table}[h]
    \caption{各極値の検定結果}
    \vspace{0.5cm}
    \centering
    \begin{tabular}{|c|c|c|c|c|c|c|c|c|c|c|}
        \hline
        \alpha = 0.05 & 1& 1& 0& 0& 0& 0& 0& 0& 1& 1 \\
        \alpha = 0.01 & 0& 1& 0& 0& 0& 0& 0& 0& 1& 0 \\
        \hline
    \end{tabular}
\end{table}
%%%%%%%%%%%%%%%%%%%%%%%%%%%%%%%%%%%%%%%%%%%%%
%%%%%%%%%%%%%%%%%%%%%%%%%%%%%%%%%%%%%%%%%%%%%
\begin{table}[h]
    \caption{極値-極値間のステップスの検定結果}
    \vspace{0.5cm}
    \centering
    \begin{tabular}{|c|c|c|c|c|c|c|c|c|c|c|}
        \hline
        \alpha = 0.05 & 1& 1& 1& 1& 0& 0& 0& 1& 1 \\
        \alpha = 0.01 & 1& 1& 1& 0& 0& 0& 0& 1& 1 \\
        \hline
    \end{tabular}
\end{table}
%%%%%%%%%%%%%%%%%%%%%%%%%%%%%%%%%%%%%%%%%%%%%

ここで改めて説明するが,私の研究では振り切った時の極小値とその後手元に戻した時の極大値,そして
それら極値間のステップ数を特徴量として有意差があるのかどうかを調べる.
\ref{fig:nakajima_takenaka}は5回スイングなので極値の特徴量は計10個,
ステップ数の特徴量は計9個である.
そしてその検定結果が表1 と表2 である.
極値を検定極値を検定した結果は有意水準が5%の時で4つ,有意水準が1%の時で2つであった.
有意差がある極値が少ないのではないかと思うかもしれないが,よく考えてほしい.
自分が振ったのに有意差が4つも出るだろうか?信頼区間を広げたとしても有意差が2つも出るだろうか?
私の考えではどんなに悪くても1つであると考えている.
そして決定的な結果が表2 である.
有意水準が1%の時でも半分以上に有意差があるという判定が出た.
これは完全に本人ではないだろう.
ここで振り方を真似する時に注目するポイントは振る速度,振り上げ速度,手首の動き,その他諸々
である.これだけのことを見ただけで真似することができるだろうか,いやできない.

以上のことから,いくら振り方を真似してもどこかがおろそかになる.
つまり完全に真似をすることは不可能である.

section{x成分の検定結果}

x成分のデータだけ検定を終えた.
結果は次のページに示す.
赤色部分は有意差が1つしかなかったものを示している.
結果より,私が真似をした時の検定と似たような結果になった.
極値検定が良くなかったときは逆にステップの極値で多くの有意差が見られ,
その逆もしかりである.
この結果からも,特徴量とするものを増やせば,なにか1つは有意差があること
が考えられる.
しかし User-E と User-F の検定結果が極値,ステップどちらも芳しくなかった.
これは2回スイングでは特徴量の数が少ないために起こったと考えられる
3回スイング以降は特徴量の数が多くなるので今後はこのような結果にはならないと
予想する.


%%%%%%%%%%%%%%%%%%%%%%%%%%%%%%%%%%%%%%%%%%%%%%%%%%%%%%%%%%%%%%%%%%%%%%%%%%%%%%%%%%%%%%%%%%%%%%%%%%%%%%%%%%%%%%%%%%
%%%%%%%%%%%%%%%%%%%%%%%%%%%%%%%%%%%%%%%%%%%%%%%%%%%%%%%%%%%%%%%%%%%%%%%%%%%%%%%%%%%%%%%%%%%%%%%%%%%%%%%%%%%%%%%%%%
\section{今後の予定}
    \begin{itemize}
        \item 検定
    \end{itemize}

    x成分では8人全員が極値を取りやすいグラフであったので問題なかった.
    しかしy成分,z成分は人によって取りやすいものと取りにくいものに分かれるので
    1つのプログラムで全員分の極値を取れない.
    もう少し粘ります.
    正直なところ,x成分だけで十分じゃない?と思っています.
    絶対加速度での検定はやめます.どの極値が何を表しているのかわかりません.

%%%%%%%%%%%%%%%%%%%%%%%%%%%%%%%%%%%%%%%%%%%%%%%%%%%%%%%%%%%%%%%%%%%%%%%%%%%%%%%%%%%%%%%%%%%%%%%%%%%%%%%%%%%%%%%%%%
%%%%%%%%%%%%%%%%%%%%%%%%%%%%%%%%%%%%%%%%%%%%%%%%%%%%%%%%%%%%%%%%%%%%%%%%%%%%%%%%%%%%%%%%%%%%%%%%%%%%%%%%%%%%%%%%%%
% \section{参考文献}
%     [1] ディープラーニングで用いられる6つの距離計算 \\
%     \url{https://sitest.jp/blog/?p=6784} \\

%     [2] MTシステムにおけるマハラノビス距離をわかりすく解説 \\
%     \url{https://navaclass.com/mahalanobis/#:~:text=%E3%83%9E%E3%83%8F%E3%83%A9%E3%83%8E%E3%83%93%E3%82%B9%E8%B7%9D%E9%9B%A2%E3%81%A8%E3%81%AF%E3%80%81%E8%A4%87%E6%95%B0,%E3%81%84%E3%82%8B%E3%81%8B%E3%82%92%E8%A1%A8%E3%81%99%E8%B7%9D%E9%9B%A2%E3%81%A7%E3%81%99%E3%80%82}\\

%     [3] 小松哲幸,平田隆幸,浪花智英,黒岩丈介,小高知宏,白井治彦,諏訪いずみ.\\
%         平成29年度修士論文 パターンロック認証に潜在する個人特徴を用いた新しい認証手法の確率.\\
%         pp. 36,37.\\

%%%%%%%%%%%%%%%%%%%%%%%%%%%%%%%%%%%%%%%%%%%%%%%%%%%%%%%%%%%%%%%%%%%%%%%%%%%%%%%%%%%%%%%%%%%%%%%%%%%%%%%%%%%%%%%%%%
%%%%%%%%%%%%%%%%%%%%%%%%%%%%%%%%%%%%%%%%%%%%%%%%%%%%%%%%%%%%%%%%%%%%%%%%%%%%%%%%%%%%%%%%%%%%%%%%%%%%%%%%%%%%%%%%%%
\clearpage
%%%%%%%%%%%%%%%%%%%%%%%%%%%%%%%%%%%%%%%%%%%%%
\begin{table}[tb]
    \caption{A\_2回スイング\_極値}
    \vspace{0.5cm}
    \centering
    \begin{tabular}{|c|c|c|c|c|}
        \hline
        User & 小1 & 大1 & 小2 & 大2 \\\hline
        B & 1&1&1&1 \\
        C & 0&1&0&1 \\
        D & 1&1&1&1 \\
        E & 1&1&1&1 \\
        F & 1&1&1&0 \\
        G & 1&1&1&1 \\
        H & 1&1&1&0 \\
        \hline
    \end{tabular}
\end{table}
%%%%%%%%%%%%%%%%%%%%%%%%%%%%%%%%%%%%%%%%%%%%%
%%%%%%%%%%%%%%%%%%%%%%%%%%%%%%%%%%%%%%%%%%%%%
\begin{table}[tb]
    \caption{B\_2回スイング\_極値}
    \vspace{0.5cm}
    \centering
    \begin{tabular}{|c|c|c|c|c|}
        \hline
        User & 小1 & 大1 & 小2 & 大2 \\\hline
        C & 1&1&1&1 \\
        D & 1&1&1&1 \\
        E & 1&1&1&1 \\
        F & 1&1&1&1 \\
        G & 1&1&1&1 \\
        H & 1&1&1&1 \\
        \hline
    \end{tabular}
\end{table}
%%%%%%%%%%%%%%%%%%%%%%%%%%%%%%%%%%%%%%%%%%%%%
%%%%%%%%%%%%%%%%%%%%%%%%%%%%%%%%%%%%%%%%%%%%%
\begin{table}[tb]
    \caption{C\_2回スイング\_極値}
    \vspace{0.5cm}
    \centering
    \begin{tabular}{|c|c|c|c|c|}
        \hline
        User & 小1 & 大1 & 小2 & 大2 \\\hline
        D & 1&1&1&1 \\
        E & 1&0&1&0 \\
        F & 1&0&1&1 \\
        G & 1&1&1&0 \\
        H & 1&0&1&1 \\
        \hline
    \end{tabular}
\end{table}
%%%%%%%%%%%%%%%%%%%%%%%%%%%%%%%%%%%%%%%%%%%%%
%%%%%%%%%%%%%%%%%%%%%%%%%%%%%%%%%%%%%%%%%%%%%
\begin{table}[tb]
    \caption{D\_2回スイング\_極値}
    \vspace{0.5cm}
    \centering
    \begin{tabular}{|c|c|c|c|c|}
        \hline
        User & 小1 & 大1 & 小2 & 大2 \\ \hline
        E & 1&1&0&0 \\
        F & 1&1&1&1 \\
        G & 1&1&0&1 \\
        H & 1&1&1&1 \\
        \hline
    \end{tabular}
\end{table}
%%%%%%%%%%%%%%%%%%%%%%%%%%%%%%%%%%%%%%%%%%%%%
%%%%%%%%%%%%%%%%%%%%%%%%%%%%%%%%%%%%%%%%%%%%%
\begin{table}[tb]
    \caption{E\_2回スイング\_極値}
    \vspace{0.5cm}
    \centering
    \begin{tabular}{|c|c|c|c|c|}
        \hline
        User & 小1 & 大1 & 小2 & 大2 \\ \hline
        F & \color{red}0&\color{red}0&\color{red}0&\color{red}1 \\
        G & \color{red}0&\color{red}1&\color{red}0&\color{red}0 \\
        H & 1&1&1&1 \\
        \hline
    \end{tabular}
\end{table}
%%%%%%%%%%%%%%%%%%%%%%%%%%%%%%%%%%%%%%%%%%%%%
%%%%%%%%%%%%%%%%%%%%%%%%%%%%%%%%%%%%%%%%%%%%%
\begin{table}[tb]
    \caption{F\_2回スイング\_極値}
    \vspace{0.5cm}
    \centering
    \begin{tabular}{|c|c|c|c|c|}
        \hline
        User & 小1 & 大1 & 小2 & 大2 \\ \hline
        G & 1&1&1&1 \\
        H & 1&0&1&0 \\
        \hline
    \end{tabular}
\end{table}
%%%%%%%%%%%%%%%%%%%%%%%%%%%%%%%%%%%%%%%%%%%%%
%%%%%%%%%%%%%%%%%%%%%%%%%%%%%%%%%%%%%%%%%%%%%
\begin{table}[tb]
    \caption{G\_2回スイング\_極値}
    \vspace{0.5cm}
    \centering
    \begin{tabular}{|c|c|c|c|c|}
        \hline
        User & 小1 & 大1 & 小2 & 大2 \\ \hline
        H & 1&1&1&1 \\
        \hline
    \end{tabular}
\end{table}
%%%%%%%%%%%%%%%%%%%%%%%%%%%%%%%%%%%%%%%%%%%%%
%%%%%%%%%%%%%%%%%%%%%%%%%%%%%%%%%%%%%%%%%%%%%%%%%%%%%%%%%%%%%%%%%%%%%%%%%%%%%%%%%%%%%%%%%%%%%%%%%%%%%%%%%%%%%%%%%%%%%%%%%%%%%%%%%%%%%%%
\clearpage
%%%%%%%%%%%%%%%%%%%%%%%%%%%%%%%%%%%%%%%%%%%%%
\begin{table}[tb]
    \caption{A\_2回スイング\_ステップ}
    \vspace{0.5cm}
    \centering
    \begin{tabular}{|c|c|c|c|}
        \hline
        User & 小1ー大1 & 大1ー小2 & 小2ー大2 \\ \hline
        B & 1&1&1 \\
        C & 1&0&1 \\
        D & \color{red}0&\color{red}1&\color{red}0 \\
        E & 1&1&1 \\
        F & 1&1&0 \\
        G & 1&1&1 \\
        H & 1&1&1 \\
        \hline
    \end{tabular}
\end{table}
%%%%%%%%%%%%%%%%%%%%%%%%%%%%%%%%%%%%%%%%%%%%%
%%%%%%%%%%%%%%%%%%%%%%%%%%%%%%%%%%%%%%%%%%%%%
\begin{table}[tb]
    \caption{B\_2回スイング\_ステップ}
    \vspace{0.5cm}
    \centering
    \begin{tabular}{|c|c|c|c|}
        \hline
        User & 小1ー大1 & 大1ー小2 & 小2ー大2 \\\hline
        C & 1&1&1 \\
        D & 1&1&1 \\
        E & 1&1&1 \\
        F & 1&1&1 \\
        G & 1&1&1 \\
        H & \color{red}0&\color{red}0&\color{red}1 \\
        \hline
    \end{tabular}
\end{table}
%%%%%%%%%%%%%%%%%%%%%%%%%%%%%%%%%%%%%%%%%%%%%
%%%%%%%%%%%%%%%%%%%%%%%%%%%%%%%%%%%%%%%%%%%%%
\begin{table}[tb]
    \caption{C\_2回スイング\_ステップ}
    \vspace{0.5cm}
    \centering
    \begin{tabular}{|c|c|c|c|}
        \hline
        User & 小1ー大1 & 大1ー小2 & 小2ー大2 \\\hline
        D & 1&1&1 \\
        E & 1&1&1 \\
        F & 1&1&1 \\
        G & 1&1&1 \\
        H & 1&1&1 \\
        \hline
    \end{tabular}
\end{table}
%%%%%%%%%%%%%%%%%%%%%%%%%%%%%%%%%%%%%%%%%%%%%
%%%%%%%%%%%%%%%%%%%%%%%%%%%%%%%%%%%%%%%%%%%%%
\begin{table}[tb]
    \caption{D\_2回スイング\_ステップ}
    \vspace{0.5cm}
    \centering
    \begin{tabular}{|c|c|c|c|}
        \hline
        User & 小1ー大1 & 大1ー小2 & 小2ー大2 \\\hline
        E & 1&1&1 \\
        F & 1&1&0 \\
        G & 1&1&0 \\
        H & 1&1&1 \\
        \hline
    \end{tabular}
\end{table}
%%%%%%%%%%%%%%%%%%%%%%%%%%%%%%%%%%%%%%%%%%%%%
%%%%%%%%%%%%%%%%%%%%%%%%%%%%%%%%%%%%%%%%%%%%%
\begin{table}[tb]
    \caption{E\_2回スイング\_ステップ}
    \vspace{0.5cm}
    \centering
    \begin{tabular}{|c|c|c|c|}
        \hline
        User & 小1ー大1 & 大1ー小2 & 小2ー大2 \\\hline
        F & \color{red}0&\color{red}0&1 \\
        G & 0&1&1 \\
        H & 1&1&1 \\
        \hline
    \end{tabular}
\end{table}
%%%%%%%%%%%%%%%%%%%%%%%%%%%%%%%%%%%%%%%%%%%%%
%%%%%%%%%%%%%%%%%%%%%%%%%%%%%%%%%%%%%%%%%%%%%
\begin{table}[tb]
    \caption{F\_2回スイング\_ステップ}
    \vspace{0.5cm}
    \centering
    \begin{tabular}{|c|c|c|c|}
        \hline
        User & 小1ー大1 & 大1ー小2 & 小2ー大2 \\\hline
        G & \color{red}0&\color{red}1&\color{red}0 \\
        H & 1&1&1 \\
        \hline
    \end{tabular}
\end{table}
%%%%%%%%%%%%%%%%%%%%%%%%%%%%%%%%%%%%%%%%%%%%%
%%%%%%%%%%%%%%%%%%%%%%%%%%%%%%%%%%%%%%%%%%%%%
\begin{table}[tb]
    \caption{G\_2回スイング\_ステップ}
    \vspace{0.5cm}
    \centering
    \begin{tabular}{|c|c|c|c|}
        \hline
        User & 小1ー大1 & 大1ー小2 & 小2ー大2 \\\hline
        H & 1&1&1 \\
        \hline
    \end{tabular}
\end{table}
%%%%%%%%%%%%%%%%%%%%%%%%%%%%%%%%%%%%%%%%%%%%%
\end{document}
