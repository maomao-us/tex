\input{texheder.tex}
\usepackage{setspace} % setspaceパッケージのインクルード
\usepackage{enumitem}
% \usepackage{graphicx}
\usepackage[dvipdfmx]{graphicx}
\usepackage{grffile}
\usepackage{url}

%「Weekly Report」
\newcommand{\Weekly}[5]{
\twocolumn[
    \begin{center}
    \bf
    第 #1 回 Weekly Report\\
    \huge
    スマートフォンのスイング動作に潜在する個人特性\\
    \LARGE
    --- フルフルによる個人認証 ---\\

    \end{center}
    \begin{flushright}
        #2 月\ \ \  #3 日 \ \ \ #4 \
        #5
    \end{flushright}
    ]
}
%\setstretch{0.5} % ページ全体の行間を設定

\begin{document}

\Weekly{7}{5}{25}{(火)}{\ 中島 基晴}


%%%%%%%%%%%%%%%%%%%%%%%%%%%%%%%%%%%%%%%%%%%%%%%%%%%%%%%%%%%%%%%%%%%%%%%%%%%%%%%%%%%%%%%%%%%%%%%%%%%%%%%%%%%%%%%%%%
%%%%%%%%%%%%%%%%%%%%%%%%%%%%%%%%%%%%%%%%%%%%%%%%%%%%%%%%%%%%%%%%%%%%%%%%%%%%%%%%%%%%%%%%%%%%%%%%%%%%%%%%%%%%%%%%%%
\section{今週までの作業内容}
    \begin{itemize}
        \item 計測データの収集
        \item 検定
        \item B4ゼミの準備
        \item 福井警察試験に向けた本格的な準備
        \item 標準化ユークリッド距離について調査
    \end{itemize}

    先週はB4ゼミの準備と福井警察試験に向けていろいろと準備をしていたので
    データの収集と検定しか行っていない.

%%%%%%%%%%%%%%%%%%%%%%%%%%%%%%%%%%%%%%%%%%%%%%%%%%%%%%%%%%%%%%%%%%%%%%%%%%%%%%%%%%%%%%%%%%%%%%%%%%%%%%%%%%%%%%%%%%
%%%%%%%%%%%%%%%%%%%%%%%%%%%%%%%%%%%%%%%%%%%%%%%%%%%%%%%%%%%%%%%%%%%%%%%%%%%%%%%%%%%%%%%%%%%%%%%%%%%%%%%%%%%%%%%%%%
\section{検定の途中結果報告}

別ページに途中結果を示す.
また,5回スイングでは改めて測定を行う前はただスイング動作を行い,改めた後は
同じスイング動作になるように意識してもらったのでその比較も現時点で比較できる人だけ示す.

%%%%%%%%%%%%%%%%%%%%%%%%%%%%%%%%%%%%%%%%%%%%%%%%%%%%%%%%%%%%%%%%%%%%%%%%%%%%%%%%%%%%%%%%%%%%%%%%%%%%%%%%%%%%%%%%%%
%%%%%%%%%%%%%%%%%%%%%%%%%%%%%%%%%%%%%%%%%%%%%%%%%%%%%%%%%%%%%%%%%%%%%%%%%%%%%%%%%%%%%%%%%%%%%%%%%%%%%%%%%%%%%%%%%%
\section{標準化ユークリッド距離について}

まずユークリッド距離とは,人が定規で測るような二点間の「通常の」距離のことであり,
ピタゴラスの公式によって与えられる.
%%%%%%%%%%%%%%%%%%%%%%%%%%%%%%%%%%%%%%%%%%%%%
\begin{equation}
    d(x, y) = d(y, x) = \sqrt{\sum_{i=1}^{n}(x_i - y_i)^2}
\end{equation}
%%%%%%%%%%%%%%%%%%%%%%%%%%%%%%%%%%%%%%%%%%%%%
これを標準化すると以下の標準化ユークリッド距離の4季を得ることができる.
%%%%%%%%%%%%%%%%%%%%%%%%%%%%%%%%%%%%%%%%%%%%%
\begin{equation}
    d(x, y) = \sqrt{\sum_{i=1}^{n}\frac{(x_i - y_i)^2}{\sigma_i}}
\end{equation}
%%%%%%%%%%%%%%%%%%%%%%%%%%%%%%%%%%%%%%%%%%%%%
ここで $\sigma_i$ は 標準偏差である.

本研究は小松哲幸さんの卒業・修士論文を参考に行っているが,
小松さんはマハラノビス距離を用いてユーザー分類を行っていた.

マハラノビス距離とはユークリッド距離にデータの相関関係を考慮した上で,
注目しているデータが中心点(平均値)からどれくらい離れているかを表す距離である.
データ値 $x$ ,平均値 $\mu$ ,標準偏差 $\sigma_i$ とした時のマハラノビス距離 $d$ は
以下の式で定義される.
%%%%%%%%%%%%%%%%%%%%%%%%%%%%%%%%%%%%%%%%%%%%%
\begin{equation}
    d = \frac{| x - \mu |}{\sigma}
\end{equation}
%%%%%%%%%%%%%%%%%%%%%%%%%%%%%%%%%%%%%%%%%%%%%

この2つの距離のどちらを使うか,またどのようにしてユーザー分類するかは未定である.

%%%%%%%%%%%%%%%%%%%%%%%%%%%%%%%%%%%%%%%%%%%%%%%%%%%%%%%%%%%%%%%%%%%%%%%%%%%%%%%%%%%%%%%%%%%%%%%%%%%%%%%%%%%%%%%%%%
%%%%%%%%%%%%%%%%%%%%%%%%%%%%%%%%%%%%%%%%%%%%%%%%%%%%%%%%%%%%%%%%%%%%%%%%%%%%%%%%%%%%%%%%%%%%%%%%%%%%%%%%%%%%%%%%%%
\section{今後の予定}
    \begin{itemize}
        \item 計測データの収集
        \item 検定
        \item 他の人同士のデータで検定
        \item 認証実験用のプログラム作成
    \end{itemize}

    3項目目の他の人同士のデータで検定に関して,データの選別で残ったデータ同士で行う検定と
    選別で省かれたデータも含めて行う検定の2種類を行おうと考えている.

%%%%%%%%%%%%%%%%%%%%%%%%%%%%%%%%%%%%%%%%%%%%%%%%%%%%%%%%%%%%%%%%%%%%%%%%%%%%%%%%%%%%%%%%%%%%%%%%%%%%%%%%%%%%%%%%%%
%%%%%%%%%%%%%%%%%%%%%%%%%%%%%%%%%%%%%%%%%%%%%%%%%%%%%%%%%%%%%%%%%%%%%%%%%%%%%%%%%%%%%%%%%%%%%%%%%%%%%%%%%%%%%%%%%%
\section{参考文献}
    [1] ディープラーニングで用いられる6つの距離計算 \\
    \url{https://sitest.jp/blog/?p=6784} \\

    [2] MTシステムにおけるマハラノビス距離をわかりすく解説 \\
    \url{https://navaclass.com/mahalanobis/#:~:text=%E3%83%9E%E3%83%8F%E3%83%A9%E3%83%8E%E3%83%93%E3%82%B9%E8%B7%9D%E9%9B%A2%E3%81%A8%E3%81%AF%E3%80%81%E8%A4%87%E6%95%B0,%E3%81%84%E3%82%8B%E3%81%8B%E3%82%92%E8%A1%A8%E3%81%99%E8%B7%9D%E9%9B%A2%E3%81%A7%E3%81%99%E3%80%82}\\

    [3] 小松哲幸,平田隆幸,浪花智英,黒岩丈介,小高知宏,白井治彦,諏訪いずみ.\\
        平成29年度修士論文 パターンロック認証に潜在する個人特徴を用いた新しい認証手法の確率.\\
        pp. 36,37.\\

%%%%%%%%%%%%%%%%%%%%%%%%%%%%%%%%%%%%%%%%%%%%%%%%%%%%%%%%%%%%%%%%%%%%%%%%%%%%%%%%%%%%%%%%%%%%%%%%%%%%%%%%%%%%%%%%%%
%%%%%%%%%%%%%%%%%%%%%%%%%%%%%%%%%%%%%%%%%%%%%%%%%%%%%%%%%%%%%%%%%%%%%%%%%%%%%%%%%%%%%%%%%%%%%%%%%%%%%%%%%%%%%%%%%%
\clearpage
%%%%%%%%%%%%%%%%%%%%%%%%%%%%%%%%%%%%%%%%%%%%%
\begin{table}[tb]
    \caption{B4_Aの検定結果}
    \vspace{0.5cm}
    \centering
    \begin{tabular}{|c|c|c|c|c|}
        \hline
        \diagbox{}{} & x & y & z & abs \\\hline
        2回 & 19 / 20 & 10 / 20 & 18 / 20 & 19 / 20 \\
        3回 & 20 / 20 & 14 / 20 & 16 / 20 & 20 / 20 \\
        4回 & 20 / 20 & 14 / 20 & 16 / 20 & 20 / 20 \\
        5回 & 15 / 20 & 11 / 20 & 15 / 20 & 20 / 20 \\
        \hline
    \end{tabular}
\end{table}
%%%%%%%%%%%%%%%%%%%%%%%%%%%%%%%%%%%%%%%%%%%%%
%%%%%%%%%%%%%%%%%%%%%%%%%%%%%%%%%%%%%%%%%%%%%
\begin{table}[tb]
    \caption{B4_Bの検定結果}
    \vspace{0.5cm}
    \centering
    \begin{tabular}{|c|c|c|c|c|}
        \hline
        \diagbox{}{} & x & y & z & abs \\\hline
        2回 & 20 / 20 & 14 / 20 & 18 / 20 & 15 / 20 \\
        3回 & 17 / 20 & 16 / 20 & 16 / 20 & 16 / 20 \\
        4回 & 19 / 20 & 19 / 20 & 19 / 20 & 18 / 20 \\
        5回 & なし & なし & なし & なし \\
        \hline
    \end{tabular}
\end{table}
%%%%%%%%%%%%%%%%%%%%%%%%%%%%%%%%%%%%%%%%%%%%%
%%%%%%%%%%%%%%%%%%%%%%%%%%%%%%%%%%%%%%%%%%%%%
\begin{table}[tb]
    \caption{B4_Cの検定結果}
    \vspace{0.5cm}
    \centering
    \begin{tabular}{|c|c|c|c|c|}
        \hline
        \diagbox{}{} & x & y & z & abs \\\hline
        2回 & 20 / 20 & 18 / 20 & 19 / 20 & 16 / 20 \\
        3回 & 20 / 20 & 20 / 20 & 18 / 20 & 13 / 20 \\
        4回 & なし & なし & なし & なし \\
        5回 & なし & なし & なし & なし \\
        \hline
    \end{tabular}
\end{table}
%%%%%%%%%%%%%%%%%%%%%%%%%%%%%%%%%%%%%%%%%%%%%
%%%%%%%%%%%%%%%%%%%%%%%%%%%%%%%%%%%%%%%%%%%%%
\begin{table}[tb]
    \caption{B4_Dの検定結果}
    \vspace{0.5cm}
    \centering
    \begin{tabular}{|c|c|c|c|c|}
        \hline
        \diagbox{}{} & x & y & z & abs \\\hline
        2回 & 20 / 20 & 20 / 20 & 20 / 20 & 20 / 20 \\
        3回 & 20 / 20 & 19 / 20 & 17 / 20 & 18 / 20 \\
        4回 & 再計測 & 再計測 & 再計測 & 再計測 \\
        5回 & 20 / 20 & 14 / 20 & 15 / 20 & 17 / 20 \\
        \hline
    \end{tabular}
\end{table}
%%%%%%%%%%%%%%%%%%%%%%%%%%%%%%%%%%%%%%%%%%%%%
%%%%%%%%%%%%%%%%%%%%%%%%%%%%%%%%%%%%%%%%%%%%%
\begin{table}[tb]
    \caption{B4_Eの検定結果}
    \vspace{0.5cm}
    \centering
    \begin{tabular}{|c|c|c|c|c|}
        \hline
        \diagbox{}{} & x & y & z & abs \\\hline
        2回 & 20 / 20 & 20 / 20 & 19 / 20 & 20 / 20 \\
        3回 & 20 / 20 & 20 / 20 & 19 / 20 & 20 / 20 \\
        4回 & 20 / 20 & 20 / 20 & 14 / 20 & 20 / 20 \\
        5回 & 18 / 20 & 20 / 20 & 14 / 20 & 20 / 20 \\
        \hline
    \end{tabular}
\end{table}
%%%%%%%%%%%%%%%%%%%%%%%%%%%%%%%%%%%%%%%%%%%%%
%%%%%%%%%%%%%%%%%%%%%%%%%%%%%%%%%%%%%%%%%%%%%
\begin{table}[tb]
    \caption{M1Mの検定結果}
    \vspace{0.5cm}
    \centering
    \begin{tabular}{|c|c|c|c|c|}
        \hline
        \diagbox{}{} & x & y & z & abs \\\hline
        2回 & 20 / 20 & 15 / 20 & 19 / 20 & 19 / 20 \\
        3回 & 18 / 20 & 17 / 20 & 18 / 20 & 19 / 20 \\
        4回 & 15 / 20 & 14 / 20 & 17 / 20 & 17 / 20 \\
        5回 & なし & なし & なし & なし \\
        \hline
    \end{tabular}
\end{table}
%%%%%%%%%%%%%%%%%%%%%%%%%%%%%%%%%%%%%%%%%%%%%
%%%%%%%%%%%%%%%%%%%%%%%%%%%%%%%%%%%%%%%%%%%%%
\begin{table}[tb]
    \caption{M1F _Aの検定結果}
    \vspace{0.5cm}
    \centering
    \begin{tabular}{|c|c|c|c|c|}
        \hline
        \diagbox{}{} & x & y & z & abs \\\hline
        2回 & 20 / 20 & 15 / 20 & 17 / 20 & 19 / 20 \\
        3回 & 20 / 20 & 19 / 20 & 12 / 20 & 20 / 20 \\
        4回 & なし & なし & なし & なし \\
        5回 & なし & なし & なし & なし \\
        \hline
    \end{tabular}
\end{table}
%%%%%%%%%%%%%%%%%%%%%%%%%%%%%%%%%%%%%%%%%%%%%
%%%%%%%%%%%%%%%%%%%%%%%%%%%%%%%%%%%%%%%%%%%%%
\begin{table}[tb]
    \caption{M1F_Bの検定結果}
    \vspace{0.5cm}
    \centering
    \begin{tabular}{|c|c|c|c|c|}
        \hline
        \diagbox{}{} & x & y & z & abs \\\hline
        2回 & 20 / 20 & 19 / 20 & 18 / 20 & 20 / 20 \\
        3回 & 20 / 20 & 20 / 20 & 18 / 20 & 20 / 20 \\
        4回 & 20 / 20 & 20 / 20 & 20 / 20 & 20 / 20 \\
        5回 & 20 / 20 & 20 / 20 & 17 / 20 & 20 / 20 \\
        \hline
    \end{tabular}
\end{table}
%%%%%%%%%%%%%%%%%%%%%%%%%%%%%%%%%%%%%%%%%%%%%

%%%%%%%%%%%%%%%%%%%%%%%%%%%%%%%%%%%%%%%%%%%%%%%%%%%%%%%%%%%%%%%%%%%%%%%%%%%%%%%%%%%%%%%%%%%%%%%%%%%%%%%%%%%%%%%%%%
%%%%%%%%%%%%%%%%%%%%%%%%%%%%%%%%%%%%%%%%%%%%%%%%%%%%%%%%%%%%%%%%%%%%%%%%%%%%%%%%%%%%%%%%%%%%%%%%%%%%%%%%%%%%%%%%%%
\clearpage
%%%%%%%%%%%%%%%%%%%%%%%%%%%%%%%%%%%%%%%%%%%%%
\begin{table}[tb]
    \caption{B4_Aの5回スイング検定結果の比較}
    \vspace{0.5cm}
    \centering
    \begin{tabular}{|c|c|c|c|c|}
        \hline
        \diagbox{}{} & x & y & z & abs \\\hline
        1回目 & 15 / 20 & 7 / 20 & 14 / 20 & 12 / 20 \\
        2回目 & 19 / 20 & 11 / 20 & 15 / 20 & 20 / 20 \\
        \hline
    \end{tabular}
\end{table}
%%%%%%%%%%%%%%%%%%%%%%%%%%%%%%%%%%%%%%%%%%%%%
%%%%%%%%%%%%%%%%%%%%%%%%%%%%%%%%%%%%%%%%%%%%%
\begin{table}[tb]
    \caption{B4_Dの5回スイング検定結果の比較}
    \vspace{0.5cm}
    \centering
    \begin{tabular}{|c|c|c|c|c|}
        \hline
        \diagbox{}{} & x & y & z & abs \\\hline
        1回目 & 12 / 20 & 15 / 20 & 10 / 20 & 13 / 20 \\
        2回目 & 20 / 20 & 14 / 20 & 15 / 20 & 17 / 20 \\
        \hline
    \end{tabular}
\end{table}
%%%%%%%%%%%%%%%%%%%%%%%%%%%%%%%%%%%%%%%%%%%%%
%%%%%%%%%%%%%%%%%%%%%%%%%%%%%%%%%%%%%%%%%%%%%
\begin{table}[tb]
    \caption{B4_Eの5回スイング検定結果の比較}
    \vspace{0.5cm}
    \centering
    \begin{tabular}{|c|c|c|c|c|}
        \hline
        \diagbox{}{} & x & y & z & abs \\\hline
        1回目 & 17 / 20 & 10 / 20 & 16 / 20 & 17 / 20 \\
        2回目 & 18 / 20 & 20 / 20 & 14 / 20 & 20 / 20 \\
        \hline
    \end{tabular}
\end{table}
%%%%%%%%%%%%%%%%%%%%%%%%%%%%%%%%%%%%%%%%%%%%%
%%%%%%%%%%%%%%%%%%%%%%%%%%%%%%%%%%%%%%%%%%%%%
\begin{table}[tb]
    \caption{M1F_Bの5回スイング検定結果の比較}
    \vspace{0.5cm}
    \centering
    \begin{tabular}{|c|c|c|c|c|}
        \hline
        \diagbox{}{} & x & y & z & abs \\\hline
        1回目 & 17 / 20 & 10 / 20 & 16 / 20 & 14 / 20 \\
        2回目 & 20 / 20 & 20 / 20 & 17 / 20 & 20 / 20 \\
        \hline
    \end{tabular}
\end{table}
\end{document}


